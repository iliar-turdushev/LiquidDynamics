\documentclass[12pt,a4paper,reqno]{amsart}
\renewcommand{\baselinestretch}{1.2}
\usepackage{amsfonts}
\usepackage{amsmath}
\usepackage{anysize}
\marginsize{2.cm}{2.cm}{1.cm}{1.cm}

\begin{document}
\begin{minipage}[b]{15cm}
\footnotesize{\emph{The V Congress of Turkic World Mathematicians,
Bulan-Sogottu, Kyrgyzstan, June 5-7, 2014} \qquad}
\end{minipage}
\bigskip

\title{ON MATHEMATICAL PROBLEM}
\maketitle

\begin{center}
Farhat Alimoglu, Jamila Akhmat kyzy \\
Dagabad (Gullarstan), Elmabad (Aralstan) \\
alim2000@tmail.com, jamila\_a\_k@fmail.tt
\end{center}
\bigskip

This mathematical problem was stated in [1]. It was partially solved
in [2]. We improve this result using the method [3].

%REFERENCES ARE NECESSARY IN THE THESIS!
%NO GRAPHICS, FIGURES!
\newtheorem{Theorem}{Theorem}
\newtheorem{Definition}{Definition}
\newtheorem{Lemma}{Lemma}
\newtheorem{Proof}{Proof}
\newtheorem{Hypothesis}{Hypothesis}

\begin{Definition}
$[3].$ If $Ax \equiv F$ then $x$ is said to be a solution of the
equation \[Ax=F.\]
\end{Definition}
We propose
\begin{Definition} If $Ax \sim F$ then $x$ is said to be a
generalized solution of the equation \[Ax=F.\]
\end{Definition}
Consider the problem
\begin{equation}
A^{p+q}x=B_{p+q}.
\end{equation}

\begin{Theorem}
  If $A \in L_{2,0}$ then the problem has a solution.
\end{Theorem}
%HINTS TO PROOFS ARE WELCOME
\begin{Proof}Uses the method of transformations [3, Chapter 2].
\end{Proof}

\begin{Theorem}If $A \in L_{2,2}$ then the problem has a
generalized solution.
\end{Theorem}

\begin{Proof}Uses the second method of transformations [3, Chapter
3].
\end{Proof}

%DO NOT WRITE: some results on the problem are obtained.
%TRY TO PRESENT CONCRETE RESULTS!

\begin{Hypothesis} If $A \in L_{2,4}$ then the generalized solution is unique.
\end{Hypothesis}

A computer program to solve the problem if $A$ is a matrix and $B$
is a vector was implemented. It gave an approximate solution.

%USING COMPUTER IS WELCOME
%FULL PAGE MUST BE

\begin{thebibliography}{2}

\bibitem{} Valiev S., Asad-zade T. (1991) New mathematical problem.
{\it {Abstracts of International conference "Mathematics and its new
applications".}} Southern University, Dagabad, pp. 64--65.

\bibitem{} Badamshin Sh., Alimoglu F. (2004) Book on mathematical problems. "Math-Science" Publishing House, Elmabad, 200 p.
\\www.mathbooks.tt/badamshinbook.htm

\bibitem{} Naryn uulu Ch. (2009) Mathematical method. {\it { Eastern Mathematical Magazine}}, vol. 9, no. 2,
pp.120--129.

\end{thebibliography}
\end{document}
