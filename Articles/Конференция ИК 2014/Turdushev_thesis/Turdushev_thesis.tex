\documentclass[12pt,a4paper,reqno]{amsart}
\renewcommand{\baselinestretch}{1.2}
\usepackage{amsfonts}
\usepackage{amsmath}
\usepackage{anysize}
\marginsize{2.cm}{2.cm}{1.cm}{1.cm}

\begin{document}
\begin{minipage}[b]{15cm}
\footnotesize{\emph{The V Congress of Turkic World Mathematicians,
Bulan-Sogottu, Kyrgyzstan, June 5-7, 2014} \qquad}
\end{minipage}
\bigskip

\title{ON CALCULATION OF INTEGRAL PARTS OF MOTION \\
IN THE THREE-DIMENSIONAL WIND FLOWS MODEL}
\maketitle

\begin{center}
Iliar Turdushev\textsuperscript{1}, Sergey Skliar\textsuperscript{1,2} \\
\textsuperscript{1}KRSU, Kyrgyzstan, Bishkek; \textsuperscript{2}AUCA, Kyrgyzstan,
 Bishkek \\
iliar.turdushev@gmail.com, sklyar51@gmail.com
\end{center}
\bigskip

In general formulation the mathematical model of wind flows of liquid in reservoir is
 described by the non-stationary initial-boundary value problem for the system of
 nonlinear equations which can be solved only using numerical methods [1]. Taking into
 account the specific of Issyk-Kul Lake the general model was simplified and some classes
 of its analytical solutions in the areas of special form were found in [2]. The
 algorithms of the model that was proposed in [2] suppose calculation of the integral
 parts $U$ and $V$ of the horizontal components of the velocity vector. For this purpose
 the next system of equations is solved:
\begin{equation}
\begin{cases}
\displaystyle\frac{\partial U}{\partial t} + \mu U - \ell V =
 -\frac{H}{\rho_0}\frac{\partial P^s}{\partial x} + \frac{\tau_x}{\rho_0}, \\[12pt]
\displaystyle\frac{\partial V}{\partial t} + \mu V + \ell U =
 -\frac{H}{\rho_0}\frac{\partial P^s}{\partial y} + \frac{\tau_y}{\rho_0}, \\[12pt]
\displaystyle\frac{\partial U}{\partial x} + \frac{\partial V}{\partial y} = 0,
 (x, y) \in \Omega_0, t > 0.
\end{cases}
\end{equation}
\begin{align}
\{(x, y) \in \partial \Omega_0\}&: U n_x + V n_y = 0, \\
t = 0&: U = U_0, V = V_0.
\end{align}

The problem (1)-(3) is considered in the two-dimensional area $\Omega_0$ that
 describes the surface of reservoir. The set $\partial \Omega_0$ is the border
 of the area $\Omega_0$.

In the system of equations (1)-(3) the next notations are used: $U = U(t, x, y)$ and
 $V = V(t, x, y)$ are integral parts of the horizontal components of velocity vector;
 $P^s = P^{s}(t, x, y)$ is a pressure on the undisturbed area $\Omega_0$; $H = H(x, y)$
 describes the bottom contour of reservoir; $\tau_x = \tau_{x}(t, x, y), \tau_y =
 \tau_{y}(t, x, y)$ are the components of the tangential stress of wind friction; $\ell =
 \ell(x, y)$ is a Coriolis force; $\rho_0$ is the mean value of density; $\mu \ge 0$
 is a parameter that defines a bottom friction; $n = (n_x, n_y)$ is a vector of an
 outer normal of the border of the area $\Omega_0$.

In this paper new projective difference schemes for solving the problem (1)-(3) are
 discussed. In order to illustrate the operation and confirm the efficiency of the
 suggested difference schemes the results of numerical experiments that were held using
 analytical solutions of the system of equations (1)-(3) found in [2] are presented.

\begin{thebibliography}{2}

\bibitem{} Marchuk G.I., Sarkisyan A.S. (1988) Matematicheskoe modelirovanie
 cirkulyacii okeana. – Nauka, Moskva, 302 p.

\bibitem{} Turdushev I.A., Skliar S.N. (2013) Analiticheskie resheniya dlya
 trehmernoi modeli vetrovyh techenii v vodoeme. {\it {Materialy 2-i mejdunarodnoi
 konferencii, posvyaschennoj 20-i letiyu obrazovaniy Kyrgyzsko-Rossiiskogo Slavyanskogo
 Universiteta i 100-letiyu professora Y. V. Bykova}}, Bishkek, Vol. 2, p. 258.

\end{thebibliography}
\end{document}
